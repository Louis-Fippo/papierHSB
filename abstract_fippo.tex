
%%%%%%%%%%%%%%%%%%%%%%% file typeinst.tex %%%%%%%%%%%%%%%%%%%%%%%%%
%
% This is the LaTeX source for the instructions to authors using
% the LaTeX document class 'llncs.cls' for contributions to
% the Lecture Notes in Computer Sciences series.
% http://www.springer.com/lncs       Springer Heidelberg 2006/05/04
%
% It may be used as a template for your own input - copy it
% to a new file with a new name and use it as the basis
% for your article.
%
% NB: the document class 'llncs' has its own and detailed documentation, see
% ftp://ftp.springer.de/data/pubftp/pub/tex/latex/llncs/latex2e/llncsdoc.pdf
%
%%%%%%%%%%%%%%%%%%%%%%%%%%%%%%%%%%%%%%%%%%%%%%%%%%%%%%%%%%%%%%%%%%%


\documentclass[runningheads,a4paper]{llncs}

\usepackage{amssymb}
\setcounter{tocdepth}{3}
\usepackage{graphicx}

\usepackage{url}
%\urldef{\mailsa}\path|{alfred.hofmann, ursula.barth, ingrid.haas, frank.holzwarth,|
%\urldef{\mailsb}\path|anna.kramer, leonie.kunz, christine.reiss, nicole.sator,|
\urldef{\mailsc}\path|{louis.fippo-fitime, olivier.roux, carito.guziolowski}@irccyn.ec-nantes.fr|    
\newcommand{\keywords}[1]{\par\addvspace\baselineskip
\noindent\keywordname\enspace\ignorespaces#1}

\begin{document}

\mainmatter  % start of an individual contribution

% first the title is needed
\title{Integrating time-series data on large-scale discrete cell-based models}

% a short form should be given in case it is too long for the running head
\titlerunning{Lecture Notes in Computer Science: Authors' Instructions}

% the name(s) of the author(s) follow(s) next
%
% NB: Chinese authors should write their first names(s) in front of
% their surnames. This ensures that the names appear correctly in
% the running heads and the author index.
%
\author{Louis Fippo Fitime\inst{1}%
\and Christian Schuster\inst{2} \and Peter Angel\inst{2}
\and Olivier Roux\inst{1}\and Carito Guziolowski\inst{1}}
%
\authorrunning{Lecture Notes in Computer Science: Authors' Instructions}
% (feature abused for this document to repeat the title also on left hand pages)

% the affiliations are given next; don't give your e-mail address
% unless you accept that it will be published
\institute{LUNAM Universit\'e, \'Ecole Centrale de Nantes, IRCCyN UMR CNRS 6597\\
(Institut de Recherche en Communications et Cybern\'etique de Nantes)\\
1 rue de la No\"e -- B.P. 92101 -- 44321 Nantes Cedex 3, France.\\
\mailsc\\
\url{http://www.irccyn.ec-nantes.fr/en/}
\and
Division of Signal Transduction and Growth Control (A100), DKFZ-ZMBH Alliance, Deutsches Krebsforschungszentrum, Heidelberg, Germany}

%
% NB: a more complex sample for affiliations and the mapping to the
% corresponding authors can be found in the file "llncs.dem"
% (search for the string "\mainmatter" where a contribution starts).
% "llncs.dem" accompanies the document class "llncs.cls".
%

\toctitle{Lecture Notes in Computer Science}
\tocauthor{Authors' Instructions}
\maketitle


\begin{abstract}
In this work we propose an automatic way of generating and verifying 
formal hybrid models of signaling and transcriptional
events, gathered in large-scale regulatory networks.This is done by 
integrating temporal and stochastic aspects of
the expression of some biological components. The hybrid approach lies 
in the fact that measurements take into account both times of 
lengthening phases and discrete switches between them. The model 
proposed is based on a real case study of keratinocytes differentiation, 
in which gene time-series data was generated upon Calcium stimulation.

To achieve this we rely on the Process Hitting (PH) formalism that was 
designed to consider large-scale system analysis.  We first propose an 
automatic way of detecting and translating biological motifs from the 
PID database to the PH formalism. Then, we propose a way of estimating 
temporal and stochastic parameters from time-series expression data of 
action on the PH. Simulations emphasize the interest of synchronizing 
concurrent events.
\keywords{time-series data, large-scale network, hybrid models, compositional approach, stochastic simulation.}
\end{abstract}


\end{document}
