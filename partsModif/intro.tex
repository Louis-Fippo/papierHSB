
\section{Introduction}

%The study and comprehension of biological systems relies on our ability to build models of such systems and
%study their properties. The way a molecular system behaves can be represented as large interaction graphs 
%composed of genes, proteins and complexes which interact to produce different dynamics in the system. 
%Thanks to the availability of gene time-series expression data, we can now observe and characterize the 
%dynamics of such components in the context of biological regulatory networks.


%These refinements were: \emph{(i)} modeling the expression of the E-cadherin signal as a discrete pulse,
%\emph{(ii)} imposing a weak synchronization of multiple regulators of a molecule, and \emph{(iii)} modeling 
%the decay of the strength of a transcription factor over the regulated gene.



% no \IEEEPARstart
Unraveling and describing the mechanisms involved in the regulation of a cell-based biological system is a fundamental 
issue. These mechanisms can be modeled as biological regulatory networks, whose analysis requires to preliminary build a 
mathematical or computational model. 
By just considering qualitative regulatory effects between components, biological regulatory networks
depict fairly well biological systems, and can be built upon public repositories such as the Pathways 
Interaction Database \cite{schaefer2009pid} and  
hiPathDB \cite{yu2012hipathdb} for human regulatory knowledge.
%the aims of the work (1first sentence & the background)
In this work we built a hybrid model of signaling and transcriptional events, gathered in large-scale regulatory networks, 
for which stochastic simulation parameters were inferred from 
gene expression time-series data.  


%NEW TEXT ============================
\modCG{
High-throughput experimental data has been used since more than one decade ago to infer 
biological regulatory models. 
A variety of methods were proposed to infer dynamic or static models of protein signaling or gene regulations
depending on the nature of the experimental data.
We can cite methods that infer static gene regulatory models from steady-state gene expression datasets of over-expression or knock-down perturbations  
using statistical models generating small-scale (ten species) models \cite{gardner2003inferring} or middle-scale (maximum 100 species) 
models \cite{Pinna2010}.
Additionally, we can cite methods that 
recovered gene regulatory dynamic models from time-series data using kinetic modeling \cite{busch2008gene,porreca2010identification} 
generating small-scale models.
Recently, static boolean models for middle and large-scale (over 100 species) 
signaling protein networks have been derived from a \emph{prior} network and 
fitted to steady-state multiple perturbation 
phosphoproteomics data \cite{guziolowski2013exhaustively,mitsos2009identifying} using combinatorial optimization through  
logic and integer linear programing to explore the vast search space of candidate boolean models.
When using time-series multi-perturbation phosphoproteomics data, results can be extended to reconstruct middle-scale dynamic signaling models  
via the use of stochastic search approaches \cite{macnamara2012state} that do not guarantee an exhaustive exploration of 
the search space of candidate models.
The approach presented in this work confronts a prior signaling and gene regulatory large-scale network, obtained from publicly curated databases, to 
time-series gene expression data, by using discrete automaton models and stochastic simulations.
Our built model verifies the agreement of expression traces over time given
a signed (activations/inhibitions), directed and cyclic prior graph. 
}


%This to come later =========================
\modCG{
The advantages and complementariness of our method with respect to the afore cited approaches are that 
it allows us to define a logic that integrates signaling and transcription events (imposing different regulatory rules on these events), 
it also integrates multi-valued states of the system components, and importantly it deals with the complexity of large-scale dynamic models.
}

%The integration of time-series data in dynamical models have been
%addressed separately by approaches that either: (a) focus first on
%modeling at small-scale the system and then on refining or improving it through the fitting with
%some data points, such as methods based on differential equations \cite{tyson2003sniffers,batt2005validation,mobashir2012simulated}, 
%(b) integrate in an efficient and complete fashion large-scale models
%and high-throughput data regardless from the system dynamics \cite{guziolowski2013exhaustively,mitsos2009identifying}, or 
%(c) fit dynamical data to middle-scale networks using an stochastic sampling of the space of behaviors and
%therefore without guarantee on finding global optima \cite{macnamara2012state}. 
%With this work we aim to fill the gaps between the aforecited methodologies.

%proposer une petite discussion sur les autres approches formelles de modélisation


%le choix que nous avons fais. Il faudra  détailler les avantages. Dire pourquoi nous choisissons le PH et pas un aute formalisme
%les grandes lignes de cette justification sont les suivantes: formalisme très expressif pour les BRN, simulation concurente 
% provenant de l'algèbre des processus, abstraction permettant de faire de analyse statiques de certaines propriétés. 
\modCG{
Several conceptually different approaches are available for modeling Biological Regulatory Network (BRN) dynamics. The most common approach is ordinary differential equations (ODE) 
<<<<<<< HEAD
that describe deterministic (population average) behavior in a continuous manner. Even for simple model including a simple interaction between two components, 
the analytical solution is impossible. Thus we must refer to simulation as the only practical method. Furthermore, continuous models require quantitative knowledge 
in terms of kinetic coefficients, which are unknown and very difficult to measure. Thereby, various abstraction approaches have been developed to make BRN models 
more convenient for analysis. Synchronous Boolean model was first proposed by Kauffman \cite{kauffman1969metabolic} and an alternative asynchronous model 
was proposed by Thomas \cite{Thomas73}. Following these two papers, many other models have been proposed \cite{snoussi1989qualitative,thieffry1995dynamical,de2003genetic,chaouiya2003qualitative} 
for modeling dynamic of BRN. All of these models are purely qualitative and discrete, thus do not incorporate  quantitative time or other quantities. 
As well, discrete models have been extended to integrate quantitative aspects.  Time aspect have been introduced by \cite{Siebert06,batt2007timed,Ahmad08,van2013timed}. 
It relies on timed automaton implemented. Anyways, this models do not take into account the stochastic aspects of the influences  of BRN.
=======
that describes deterministic (population average) behaviour in a continuous maner. Even for simple models including a simple interaction between two components, 
the analytical solution is impossible. Thus we must refer to simulation as the only practical method. Futhermore, continuous models require quantitative knowledge 
in terms of kinetic coefficients, which are unknown and very difficult to measure. Thereby, various abstraction approches have been developed to make BRN models 
more convenient for analyses. The synchronous Boolean model was first proposed by Kauffman \cite{kauffman1969metabolic} and an alternative asynchronous model 
was proposed by Thomas \cite{Thomas73}. Folowing these two papers, many other models have been proposed \cite{snoussi1989qualitative,thieffry1995dynamical,de2003genetic,chaouiya2003qualitative} 
for modeling the dynamic of a BRN. All of these models are purely qualitative and discrete, thus do not incorporate  quantitative time or other quantities. 
Therefore, discrete models have been extended to integrate quantitative aspects, such as time, which was introduced by \cite{Siebert06,batt2007timed,Ahmad08,van2013timed}. 
These approaches rely on timed automata implementations. These models, however, do not take into acount the stochastic aspects of the influences of a BRN.
>>>>>>> 12c241a1d88dedd17eca66bd84aa7ba99f981d1c
}



In the context of modeling and analyzing stochastic and concurrent biological systems various formalisms have been introduced such as 
Stochastic Petri Nets which is suitable for the representation of parallel systems \cite{molloy1982performance}. 
They have been successfully applied in many areas; in particular, the specification of Petri Nets
allows an accurate modeling of a wide range of systems including biological systems \cite{heiner2008petri}. The major 
problem of Stochastic Petri Nets is that, generally, they do not lead to compact models. In addition,
they do not provide results to deal with the state space explosion and are thus computationally
expensive when modeling large-scale biological networks. 
%TODO
The Stochastic pi-calculus formalism was introduced by \cite{priami1995stochastic} and used in 
\cite{maurin2009modeling} for the modeling of biological systems. Stochastic pi-calculus has a rich
expressiveness and is well adapted for the use of compositional approach.
In this work we rely on this formalism through the Process Hitting (PH) framework \cite{PMR10-TCSB}, 
since it is especially useful for studying systems composed of biochemical interactions, and provides
stochastic simulation as well as efficient algorithms, \modCG{based on the verification of state reachability}, to study dynamical properties of the system.
The PH framework uses qualitative and discrete information of the system without requiring enormous parameter estimation tasks
 for its stochastic simulation. 
This framework has been previously used to verify dynamical properties on biological systems without integrating high-throughput experimental data.

In this work we provide a method to build a time-series data integrated PH model and we evaluate 
the prediction power of this model concerning the simultaneously predicted traces of 12 gene components of the system upon system stimulation.
%So far, this method has been successfully demonstrated only on very well-known systems and without exploiting 
%high-throughput measures. We believe, however, that the use of high-throughput data has become unavoidable with 
%the advent of massive, publicly available data sets in the form of well-standardized DNA microarray data and, 
%more recently, in the form of phospho-proteomics data.  
% les contributions du travail que nous présentons: Génération automatique des modèles en PH(avec et 
% sans synchronisation), Estiamtion des paramètres des times series data Et intégration des paramètres 
% dans le modèle.
%TODO revoir le temps et les liaisons
The main  results of this work are: (1) automatic generation of PH models integrating gene transcription and signaling events, 
with and without synchronization of concurrent events, from the Pathways Interaction Database, 
(2) parameter estimation from time-series data and parameter integration in the PH model, and
(3) comparison of the PH model predictions and experimental results.
To illustrate our approach, we used a time-series dataset of human keratinocytes cells, which shows
 the fluctuations of \modCG{mRNA expression} across time upon Calcium stimulation.
This dataset was built to study keratinocytes differentiation, a time-dependent
 process in which the sequence of activation of signaling proteins is not yet completely understood.
The method proposed in this paper remains general and can be applied to other case-studies.

%first, we built an interaction graph linking a signaling molecule, 
%E-cadherin (Calcium sensitive protein), to genes present in our time-series data and to key cellular processes for our case study, such as 
%keratinocyte-differentiation and cellular-proliferation.  This graph was automatically extracted from PID. Second, we propose an automatic                                     %ici il manque une reference.
%transformation of selected known biological patterns present in  PID in order to generate PH modules;
% adding necessary constraints to the PH model to avoid oscillations.  
%Third, we propose a way of estimating temporal and stochastic parameters from time-series expression data to 
%model the measured genes.  These parameters are used for the stochastic simulation of the model.  Finally, 
%we discretized the experimental data to allow the comparison with simulation results for the above mentioned case 
%study analysis. 
% You must have at least 2 lines in the paragraph with the drop letter
% (should never be an issue)


%\hfill mds
 
%\hfill March 20, 2015
