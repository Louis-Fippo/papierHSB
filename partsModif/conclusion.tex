\section{Conclusion}
This work describes the steps towards the integration of time-series data in large-scale cell-based models. 
We proposed an automatic method to build a timed and stochastic PH model from pathways of biochemical reactions present in 
the Pathway Interaction Database (PID). 
%
%<<<<<<< HEAD
%As a case-study, we built a model combining signaling and transcription events relevant to keratinocyte differentiation induced by Calcium, which linked E-cadherin nodes and $12$ genes, which 
%expression profiles was measured upon Calcium stimulation over time. The interaction graph represented by the model had $293$ and $375$ edges.
%=======
As a case-study we built a model combining signaling and transcription events relevant to keratinocyte differentiation induced by Calcium, which linked E-cadherin nodes and $12$ genes, which 
expression profiles was measured upon Calcium stimulation over time. The interaction graph represented by the model had $293$ nodes and $375$ edges.
>>>>>>> 12c241a1d88dedd17eca66bd84aa7ba99f981d1c
%
We proposed a method to discretize time-series gene expression data, so they can be integrated to the PH simulations and logically explained by the PH stochastic analyses. 
%
Additionally, we implemented a method to automatically estimate the temporal and stochastic
parameters for the PH simulation, so this estimation process will not be biased by over fitting. 
%

Our results show that  we can observe dynamic effects on $11$ out of $12$ genes, for which $5$ of them represent accurate predictions, and $6$ of them missed few dynamic levels.
This error may be also a result from the incompleteness of the regulatory information in PID.
Moreover, when observing the predicted behavior of biological processes linked to Calcium stimulation, our predictions agreed with experimental and literature-based evidences.
%this reproduce accuretely $9$ out of $12$ of the genes, 
Overall, with this work we show the feasibility of modeling and simulating large-scale networks with very few parameter estimation 
and having good quality predictions.
%, (3) we can reproduce tendences of the dynamic of components
%and take into accout the stochastic and time aspect of the behaviors of the biological system.

As perspectives of this work we intend to study the effects of computing automatically the concurrent rules on this system.
Also, we intend to improve the model prediction quality by empirically obtaining the dynamics of the system components by performing large stochastic simulations, as well 
as by implementing static analysis of quantitative properties by adding probabilistic features to the PH static solver.

