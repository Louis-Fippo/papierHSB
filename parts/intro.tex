
\section{Introduction}

%The study and comprehension of biological systems relies on our ability to build models of such systems and
%study their properties. The way a molecular system behaves can be represented as large interaction graphs 
%composed of genes, proteins and complexes which interact to produce different dynamics in the system. 
%Thanks to the availability of gene time-series expression data, we can now observe and characterize the 
%dynamics of such components in the context of biological regulatory networks.


%These refinements were: \emph{(i)} modeling the expression of the E-cadherin signal as a discrete pulse,
%\emph{(ii)} imposing a weak synchronization of multiple regulators of a molecule, and \emph{(iii)} modeling 
%the decay of the strength of a transcription factor over the regulated gene.



% no \IEEEPARstart
The comprehension of the mechanisms involved in the regulation of a cell-based biological system is a fundamental 
issue. These mechanisms can be modeled as biological regulatory networks, which analysis requires to preliminary build a 
mathematical or computational model. 
By just considering qualitative regulatory effects between components, biologic regulatory networks
depict fairly well biological systems, and can be built upon public repositories such as the Pathways 
Interaction Database \cite{schaefer2009pid} and  
hiPathDB \cite{yu2012hipathdb} for human regulatory knowledge.

%the aims of the work (1first sentence & the background)

In this work we built an hybrid model of signaling and transcriptional events, gathered in large-scale regulatory networks, 
which stochastic simulation parameters were inferred from 
gene expression time-series data.  
The integration of time-series data in large-scale dynamical models have been
addressed separately by approaches that either: (a) focus first on
modeling at small-scale the system and then on refining or improving it through the fitting with
some data points, such as methods based on differential equations \cite{tyson2003sniffers,batt2005validation,mobashir2012simulated}, 
(b) integrate in an efficient and complete fashion large-scale models
and high-throughput data regardless from the system dynamics \cite{guziolowski2013exhaustively,mitsos2009identifying}, or 
(c) fit dynamical data to middle-scale networks using an stochastic sampling of the space of behaviors and
therefore without guarantee on finding global optima \cite{macnamara2012state}. 
With this work we aim to fill the gaps between the aforecited methodologies.

%proposer une petite discussion sur les autres approches formelles de modélisation


%le choix que nous avons fais. Il faudra  détailler les avantages. Dire pourquoi nous choisissons le PH et pas un aute formalisme
%les grandes lignes de cette justification sont les suivantes: formalisme très expressif pour les BRN, simulation concurente 
% provenant de l'algèbre des processus, abstraction permettant de faire de analyse statiques de certaines propriétés. 

In the context of modeling and analyzing stochastic and concurent biological systems various formalisms have been introduced such as 
Stochastic Petri Nets which is suitable for the representation of parallel systems \cite{molloy1982performance}. 
They have been successfully applied in many areas; in particular, the specification of Petri Nets
allows an accurate modeling of a wide range of systems including biological systems \cite{heiner2008petri}. The major 
problem of Stochastic Petri Nets is that, generally, they do not lead to compact models. In addition,
they do not provide results to deal with the state space explosion and are thus computationally
expensive when modeling large-scale biological networks. 

%TODO
The Stochastic pi-calculus formalism was introduced by \cite{priami1995stochastic} and used in 
\cite{maurin2009modeling} for the modeling of biological systems. Stochastic pi-calculus has a rich
expressivity and is well adapted for the use of \modCG{compositionnal approach}. \modCG{[What do you mean by compositionnal approach? Is it with ``nn'' ?]}
In this work we rely on this formalism through the Process Hitting (PH) framework \cite{PMR10-TCSB}, 
since it is especially useful for studying systems composed of biochemical interactions, and provides
stochastic simulation as well as efficient static methods to model dynamical properties of the system.
The PH framework uses qualitative and discrete information of the system without requiring enormous parameter estimation tasks
 for its stochastic simulation. 
This framework has been previously used to verify dynamical properties on biological systems without integrating high-throughput experimental data.
In this work we provide a method to build a time-series data integrated PH model and we evaluate 
the prediction power of this model concerning the simulatenously predicted traces of several components of the system upon system stimulation.
More precisely,
%So far, this method has been successfully demonstrated only on very well-known systems and without exploiting 
%high-throughput measures. We believe, however, that the use of high-throughput data has become unavoidable with 
%the advent of massive, publicly available data sets in the form of well-standardized DNA microarray data and, 
%more recently, in the form of phospho-proteomics data.  
% les contributions du travail que nous présentons: Génération automatique des modèles en PH(avec et 
% sans synchronisation), Estiamtion des paramètres des times series data Et intégration des paramètres 
% dans le modèle.
%TODO revoir le temps et les liaisons
the main  results of this work are: (1) automatic generation of PH models, with and without synchronisation gates, from the Pathways Interaction Database, 
(2) parameter estimation from time-series data and parameter integration in the PH model, and
(3) comparison of the PH model predictions and experimental results.

%first, we built an interaction graph linking a signaling molecule, 
%E-cadherin (Calcium sensitive protein), to genes present in our time-series data and to key cellular processes for our case study, such as 
%keratinocyte-differentiation and cellular-proliferation.  This graph was automatically extracted from PID. Second, we propose an automatic                                     %ici il manque une reference.
%transformation of selected known biological patterns present in  PID in order to generate PH modules;
% adding necessary constraints to the PH model to avoid oscillations.  
%Third, we propose a way of estimating temporal and stochastic parameters from time-series expression data to 
%model the measured genes.  These parameters are used for the stochastic simulation of the model.  Finally, 
%we discretized the experimental data to allow the comparison with simulation results for the above mentioned case 
%study analysis. 
% You must have at least 2 lines in the paragraph with the drop letter
% (should never be an issue)


%\hfill mds
 
%\hfill March 20, 2015
