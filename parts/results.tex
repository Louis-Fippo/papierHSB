
\section{Results}

\subsection{From PID to PH: An automatic generation of PH code of the network}
%give statistics

To simulate of the model we generated a PINT code to be simulated by the PINT simulator \footnote{Available at \url{http://process.hitting.free.fr}}. 
PINT implements static analyses for computing dynamical properties on very large-scale Automata Networks.
For the PINT code generation we use two procedures as described in the method section. The first procedure that detects the patterns and its code; the second 
procedure that generates the PINT code by taking in consideration the type of the pattern to better refine the dynamic of the system from a structural point of view.
This refinement is done by introducing the synchronization sort  which is a generalization of the cooperation sort. This allows  to avoid artificial oscillations
in the dynamics of the components of the system.




\subsection{Simulation}

We simulated the model with and without the inclusion of the synchronization sort. In the following, we present the results of 
the simulation.

\subsubsection{Without the introduction of synchronization sort}
One can easily notice in \pref{fig:rwos} the occurence of oscillations. This is not the expected behavior from the biological system
but it is coherent with the choice of the modeling and the way the simulator works as explained in \ref{sssec:synchronization}: cooperation
sorts are used to model multiple regulators of a common target
where this is clearly identifiable. In other cases we left the components act independently.
It is important to notice that the intensity of the oscillation is linked with 
the size of the concurrence, i.e. the number of predecessors that a node in the network has.
Despite the presence of the oscillations, the model reproduces expected dynamical behavoirs  namely
the dynamic of components, the signal transduction and takes into account the stochastic and time aspect of the model.

\begin{figure}[!ht]
\centering
\includegraphics[width=5in,height=3in]{images/resultWOS.pdf}
\caption{\bf Results of simulations without introducing the synchronization sort. In black line the expected behaviors
come from the discretization of time-series data. In Blue line the simulation behavior.}
\label{fig:rwos}
\end{figure}
\subsubsection{With the introduction of synchronization sort}

From \pref{fig:rws} we can see that the introduction of the synchronization sort significatively reduces the 
impact of concurrence by the introduction of the synchronization. The result shows  a 
clear elimination of the previously observed oscillation in \pref{fig:rwos}. One can see that the expression levels of genes \emph{uPAR}, \emph{MKP1},
\emph{IL1 beta} are well reproduced, while \emph{Hes5} and \emph{IL8} are not activated. This result can be observed when the activation
signal has not been able to propagate through the network due to the non determinism  and concurrency.


\begin{figure}[!ht]
\centering
\includegraphics[width=5in,height=3in]{images/resultWS.pdf}
\caption{\bf Results of simulations by introducing the synchronization sort. In black line the expected behaviors
come from the discretization of time-series data. In Blue line the simulation behavior.}
\label{fig:rws}
\end{figure}



% An example of a floating figure using the graphicx package.
% Note that \label must occur AFTER (or within) \caption.
% For figures, \caption should occur after the \includegraphics.
% Note that IEEEtran v1.7 and later has special internal code that
% is designed to preserve the operation of \label within \caption
% even when the captionsoff option is in effect. However, because
% of issues like this, it may be the safest practice to put all your
% \label just after \caption rather than within \caption{}.
%
% Reminder: the "draftcls" or "draftclsnofoot", not "draft", class
% option should be used if it is desired that the figures are to be
% displayed while in draft mode.
%
%\begin{figure}[!t]
%\centering
%\includegraphics[width=2.5in]{myfigure}
% where an .eps filename suffix will be assumed under latex, 
% and a .pdf suffix will be assumed for pdflatex; or what has been declared
% via \DeclareGraphicsExtensions.
%\caption{Simulation Results}
%\label{fig_sim}
%\end{figure}

% Note that IEEE typically puts floats only at the top, even when this
% results in a large percentage of a column being occupied by floats.


% An example of a double column floating figure using two subfigures.
% (The subfig.sty package must be loaded for this to work.)
% The subfigure \label commands are set within each subfloat command, the
% \label for the overall figure must come after \caption.
% \hfil must be used as a separator to get equal spacing.
% The subfigure.sty package works much the same way, except \subfigure is
% used instead of \subfloat.
%
%\begin{figure*}[!t]
%\centerline{\subfloat[Case I]\includegraphics[width=2.5in]{subfigcase1}%
%\label{fig_first_case}}
%\hfil
%\subfloat[Case II]{\includegraphics[width=2.5in]{subfigcase2}%
%\label{fig_second_case}}}
%\caption{Simulation results}
%\label{fig_sim}
%\end{figure*}
%
% Note that often IEEE papers with subfigures do not employ subfigure
% captions (using the optional argument to \subfloat), but instead will
% reference/describe all of them (a), (b), etc., within the main caption.


% An example of a floating table. Note that, for IEEE style tables, the 
% \caption command should come BEFORE the table. Table text will default to
% \footnotesize as IEEE normally uses this smaller font for tables.
% The \label must come after \caption as always.
%
%\begin{table}[!t]
%% increase table row spacing, adjust to taste
%\renewcommand{\arraystretch}{1.3}
% if using array.sty, it might be a good idea to tweak the value of
% \extrarowheight as needed to properly center the text within the cells
%\caption{An Example of a Table}
%\label{table_example}
%\centering
%% Some packages, such as MDW tools, offer better commands for making tables
%% than the plain LaTeX2e tabular which is used here.
%\begin{tabular}{|c||c|}
%\hline
%One & Two\\
%\hline
%Three & Four\\
%\hline
%\end{tabular}
%\end{table}


% Note that IEEE does not put floats in the very first column - or typically
% anywhere on the first page for that matter. Also, in-text middle ("here")
% positioning is not used. Most IEEE journals/conferences use top floats
% exclusively. Note that, LaTeX2e, unlike IEEE journals/conferences, places
% footnotes above bottom floats. This can be corrected via the \fnbelowfloat
% command of the stfloats package.
